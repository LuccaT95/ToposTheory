\documentclass[a4paper, 11pt]{article}

%basic
\usepackage[utf8]{inputenc}

% for handeling citations
\usepackage[backend=biber, style=numeric, citestyle = numeric]{biblatex}  
\addbibresource{bibFile.bib} 

%for mathematics
\usepackage{mathtools}
\usepackage{amsfonts}
\usepackage{amsthm}
\usepackage{amssymb}

% Mathematical environments
\newtheorem{theorem}{Theorem}
\newtheorem{proposition}[theorem]{Proposition}
\newtheorem{corollary}[theorem]{Corollary}
\newtheorem{lemma}[theorem]{Lemma}
\newtheorem{remark}[theorem]{Remark}

\theoremstyle{definition}
\newtheorem{definition}[theorem]{Definition}
\newtheorem*{observation}{Observation}
\newtheorem{example}{Example}

% category theory notation
\newcommand{\cat}{%
	\mathbf %
}

\newcommand{\E}[1]{E \, (#1)}

\newcommand{\domain}[ 1 ]{%
	\mathrm{dom} \, #1 %
}

\newcommand{\codomain}[ 1 ]{%
	\mathrm{cod} \, #1%
}

\newcommand{\notion}[1]{\text{#1 }}

\newcommand{\commSquare}[4]{\notion{commSquare} (#1, \, #2, \, #3, \, #4)}

\newcommand{\product}[5]{\notion{product} (#1, \, #2, \, #3, \, #4, \, #5)}
\newcommand{\expon}[5]{\notion{exponential} (#1, \, #2, \, #3, \, #4, \, #5)}

\title{Topos Theory in Free Logic}

\author{Lucca Tiemens}

\begin{document}

\maketitle

Let me start by stating the axioms for a category again for the sake of completeness.

\begin{definition}

	A structure $\cat C = (C, dom, cod, \cdot)$ is called a category if and only if it satisfies the following axioms where $x, y$ and $z \in \cat C$ :
	\begin{itemize}
		\item $\E{\domain{x}} \longrightarrow \E{x}$
		\item $\E{\codomain{y}} \longrightarrow \E{y}$
		\item $\E{x\cdot y} \longleftrightarrow \domain{x} \simeq \codomain{y}$
		\item $x \cdot (y \cdot z) \cong (x \cdot y) \cdot z$
		\item $x \cdot (\domain{x}) \cong x$
		\item $(\codomain{y} \cdot y) \cong y.$ 
	\end{itemize}

\end{definition} 

\section{Basic Definitions}

In the following let $\cat C$ always denote a category. We have the convenient abbreviation \emph{type} denoting an \emph{existing} identity arrow.

\begin{definition}[type]
	$x \in \cat C$ is called a type if and only if \[x \simeq \domain{x}\] 
\end{definition}

\noindent Furthermore, we adopt notation from category theory for \emph{existing} arrows.

\begin{definition}[existing arrow]
	For $x,a,b \in \cat C$ writing $x:a \rightarrow b$ abbreviates 
	\[ \domain{x} \simeq a \wedge \codomain{x} \simeq b \]
\end{definition}

\noindent For an arrow which does not necessarily exist, we also introduce notation.

\begin{definition}[general arrow]
	For $x,a,b \in \cat C$ writing $x:a \Rightarrow b$ abbreviates 
	\[ \domain{x} \cong a \wedge \codomain{x} \cong b \]
\end{definition}

\noindent A commutative square is easy to write down. The biggest challenge is to remember the order in which the arguments are stated. I have adopted the habit of starting at the lower right corner and then moving counter-clockwise.

\begin{definition}[commutative square]
	For $f, g, p, q \in \cat C$ we define
	\[\commSquare{g}{p}{q}{f} \Longleftrightarrow  g \cdot p \cong f \cdot q \]
\end{definition} 

Note, that we do not explicitly need to say what the domains and codomains are. This can be inferred by using the functions \emph{dom} and \emph{cod}.

At this point I will recall the definition of a product. $p1$ and $p2$ should be thought of as the projection maps. Also note, that $\forall_f$ ($\exists_f$) denotes the \emph{free} universal (existential) quantifier.

\begin{definition}
	For $a, b, c, p1, p2 \in \cat C$ we define
	\begin{align*}
		\product{a}{b}{c}{p1}{p2} \Longleftrightarrow \quad & p1:c\Rightarrow a \wedge p2:x\Rightarrow b \, \wedge \\
		& \forall_f x f g. \, (f:x\rightarrow a \wedge g:x\rightarrow b) \longrightarrow \\
		& \exists_f!h. \, (h:x\rightarrow c \wedge f \simeq p1 \cdot h \wedge g \simeq p2 \cdot h)
	\end{align*}  
\end{definition}

\section{Category with binary products}

We are now in the position to write down the axioms for a category that has binary products. Taking the product between two elements of a category is defined as a primitive operation. It is defined between types as well as between arrows. Furthermore, the projection arrows are skolemized.

\begin{definition}
	A category with binary products is a category $\cat C$ together with maps $\otimes: \cat C \times \cat C \to \cat C$, $p1: \cat C \to \cat C$ and $p2: \cat C \to \cat C$ such that 
	\begin{enumerate}
		\item $\E{a \otimes b} \longrightarrow \E{a} \wedge \E{b}$
		\item $\E{p1(a)} \longrightarrow \E{a}$
		\item $\E{p2 (b)} \longrightarrow \E{b}$
		\item $\notion{type} a \, \wedge \notion{type} b \, \longrightarrow \product{a}{b}{(a\otimes b)}{(p1(a \otimes b))}{(p2 (a \otimes b))}$
		\item $\commSquare{a}{p1(\domain{a} \otimes \domain{b})}{a \otimes b}{p1(\codomain{a} \otimes \codomain{b})}$
		\item $\commSquare{b}{p2(\domain{a} \otimes \domain{b})}{a \otimes b}{p2(\codomain{a} \otimes \codomain{b})}$
	\end{enumerate}
\end{definition}

\noindent Axiom 4 makes $a \otimes b$ the product of $a$ and $b$. Axioms 5 and 6 enable us to form the product between arrows as well. Let $a:x\rightarrow y$ and $b:v \rightarrow w$ be arrows, $a\otimes b$ then denotes the unique arrow $<a, b>:x\otimes v \rightarrow y\otimes w$. 

Note, that this definition does not force existence of elements. There are models of only non-existing elements - the easiest one being just one non-existing element. Therefore, we have not excluded the empty category in our axiomatization.


\section{Cartesian Category}

The next particular category I want to axiomatize is a Cartesian category. By such a category I mean a category with \emph{binary} products, that has a \emph{final} type as well as \emph{equalizers}. I will define the required notions below.
A different way to view a Cartesian category is thus a category with all \emph{finite} limits.
However, I have not formulated the notion of a general limit in our formalism. It seems to me that this will not work well with theorem proving because of the \emph{cone} constructions involved. 
Do you think we need the general definition for Topos Theory or are we okay with using finite products and equalizers?

Let's get started with the definition of a final type. We did this before, however, there was a small mistake in the original formulation. The correct definition is

\begin{definition}[final type] 
	Let $z \in \cat C$. We then define
	\[ \notion{final} z \Longleftrightarrow \forall_f t. \, (\notion{type} t) \longrightarrow (\exists_f!f. \, f:t\rightarrow z)\]
\end{definition} 

In the earlier definition the condition of $t$ being a type was missing. Without this condition a category with a final type does not have any arrows except identity-arrows.

\begin{definition}[equalizer between two arrows]
	Let $f, g, e \in \cat C$. Then we define 
	\begin{align*}
		\notion{equalizer}(f, g, e) & \Longleftrightarrow f \cdot e \simeq g \cdot e \, \wedge \\
		& \forall_f z. \, (f \cdot z \simeq g \cdot z) \longrightarrow (\exists_f!u. \, u:\domain{z}\rightarrow\domain{e} \, \wedge \, e \cdot u \simeq z)
	\end{align*}
\end{definition}

Note that if $f$ and $g$ are not parallel, i.e. they do not agree on the domain or on the codomain, then $\notion{equalizer}(f,g,e)$ will always be false and hence there is no equalizer.

Now we are in the position to formulate a Cartesian category.

\begin{definition}[Cartesian category]
	A Cartesian category $ \cat C$ is a category with binary products, maps, called $\doteq \, :\cat C \times \cat C \to \cat C$ and $!_1:\cat C \to \cat C$, and a constant $\boldsymbol{1} \in \cat C$ such that
	\begin{enumerate}
		\item $(f:\domain{g} \rightarrow \codomain{g}) \longrightarrow \notion{equalizer}(f, g, (f \doteq g))$
		\item $\notion{final} \boldsymbol{1}$
		\item $\notion{type} t \longrightarrow (!_1 \, t):t\rightarrow1$
		\item $E (!_1 \, t) \longrightarrow \notion{type} t$
		\item $E (f \doteq g) \longrightarrow f:\domain{g} \rightarrow \codomain{g}$
	\end{enumerate}
\end{definition}

\section{Exponential Category}

Now that we have limits, the next construction we need before we can axiomatize a Topos are exponential types. Because we can use the $\otimes$ operation on all arrows, an exponential is defined as usual in the literature, except that we skolemize the transpose map.

\begin{definition}[exponential between types]
	Let $a, b, c, \epsilon \in \cat C$ where $\cat C$ is a category with binary products. Furthermore, let $tp: \cat C \to \cat C$ be a map. We define
	\begin{align*}
		\expon{a}{b}{c}{\epsilon}{tp} & \Longleftrightarrow \epsilon:(a \otimes c) \Rightarrow b \, \wedge \\
		& \forall_f z \, f. \, (f:(z \otimes a) \rightarrow b \; \longrightarrow \\ 
		&((tp \, f):z \rightarrow c \, \wedge \, (\epsilon \cdot \domain{a} \otimes (tp \, f)) \simeq f \, \wedge \\
		& (\forall_f \hat{f}. \, (\hat{f}:z \rightarrow c \, \wedge \, (\epsilon \cdot \domain{a} \otimes \hat{f}) \simeq f) \longrightarrow \hat{f} \simeq (tp \, f))))
	\end{align*}
	We call $\epsilon$ the \emph{evaluation map} and $\text{tp}$ the \emph{transpose map}.
\end{definition}

With this definition in hand we can define an exponential category. By this I mean a category for which we have an exponential between two \emph{types}.

As far as I have seen it is not necessary to enable exponential between \emph{arrows} because we can define the maps between exponential types in terms of ... . Hence we should be able to prove that exponentiation by a type is a functor without further axiomatization.

\begin{definition}[Exponential category]
	An Exponential category $\cat C$ is a category with binary products and maps that are called $(\bullet)^{\bullet}: \cat C \times \cat C \to \cat C$, $\epsilon: \cat C \times \cat C \to \cat C$ and $\text{tp}: \cat C \to \cat C$ such that
	\begin{enumerate}
		\item $\notion{type} a \, \wedge \, \notion{type} b \longrightarrow \expon{a}{b}{b^{a}}{\epsilon}{\text{tp}}$
		\item $E \, b^{a} \longrightarrow E \, a \, \wedge \, E\, b $
		\item $E \, \epsilon(a, b) \longrightarrow E \, a \, \wedge \, E \, b$
		\item $E \, (tp\, a) \longrightarrow E \, a$
	\end{enumerate}
\end{definition}

\end{document}


















